\documentclass[11pt]{article}

\usepackage[margin=1in]{geometry}
\usepackage{fancyhdr}
\pagestyle{fancy}
\usepackage{graphicx}
\usepackage{adjustbox}
\usepackage{listings,lstautogobble}
\usepackage{courier}

\lstset{basicstyle=\footnotesize\ttfamily,
	breaklines=true,
	autogobble=true,
	language=SQL}


\lhead{CS 351 Assignment \#3 }
\chead{John Karasev}
\rhead{March 2, 2018}


\begin{document}

\section{Data}

        \adjustbox{valign=t}{\includegraphics[width=15cm]{./figures/dbfig.pdf}}
	\vspace{2cm}\\

	Above are all the schemas and their relationships with other schemas.
	PK stands for primary key and FK stands for foreign key. All the
	entries in the movie csv that where not atomic where broken down into
	two separate tables. Since atomic columns like \textbf{genre} and
	\textbf{production\_companies} had a many-to-many relationship with the
	movie table, the first table was the relationship table and the
	second table contained specific information like genre id and genre
	name. All tables are in BCNF because (1) all attributes are atomic, (2)
	all non-key attributes depend only on the candidate key, (3)
	and in all $X \rightarrow Y$ FD's, $X$ is a candidate key. For example,
	all attributes in the \textbf{movie} table only depend on the
	candidate key id and all attributes in  \textbf{movie\_genre} table
	only depend on the candidate key id. This satisfies BCNF condition
	because the attribute on the right side is a candidate key.

\pagebreak

Refer to figure above for table attributes, primary key, and foreign keys.

\begin{enumerate}
	\item{\textbf{movie :}}
		Contains all the columns that where atomic in the csv file.
		This table is in BCNF because in the \textbf{movie} FD
		$X \rightarrow A,B,C,D,E,F,G,H,I,J,K,L,M,N$
		the left side $X$ is a candidate key \textbf{id}. This satisfies
		the BCNF condition.
	\item{\textbf{movie\_genre :}}
		Relationship table between many-to-many relationship of
		\textbf{genre} and \textbf{movie}. Contains a unique
		\textbf{id}, \textbf{movie\_id}, and \textbf{genre\_id} where
		\textbf{movie\_id} and \textbf{genre\_id} reference \textbf{id}
		in \textbf{id} in \textbf{movie} and \textbf{id} in
		\textbf{id} in \textbf{genre} respectively. This table is
		in BCNF because in the FD $X \rightarrow Y,Z$ $X$ is the
		candidate key. This satisfies the BCNF condition.
	\item{\textbf{genre :}}
		Contains all the genre names and their id that where found
		in the genre column from all rows. \textbf{movie\_genre}
		references this table. Also in BCNF because the FD
		$X \rightarrow Z$ satisfies BCNF condition because $X$ is a
		candidate key.
	\item{\textbf{spoken\_languages :}}
		Contains all the spoken languages id's and names that were
		encountered in the csv. Also in BCNF, see \textbf{genre} for
		explanation.
	\item{\textbf{movie\_spoken\_languages :}}
		Relationship table for the many-to-many relationship between
		\textbf{spoken\_languages} and \textbf{movie}. In BCNF. See
		\textbf{movie\_genre} for explanation.
	\item{\textbf{keywords :}}
		Contains all keyword names and id's encountered in csv.
		In BCNF. See \textbf{genre} for explanation.
	\item{\textbf{movie\_keywords :}}
		All keywords and their repective id's that where encountered in
		the csv where stored here. In BCNF. See
		\textbf{movie\_genre} for explanation.
	\item{\textbf{production\_countries :}}
		Contains the countries id and names that where encountered in
		the csv. In BCNF, see \textbf{genre :} for explanantion.
	\item{\textbf{movie\_production\_countries :}}
		Relationship table for the many-to-many relationship between
		\textbf{movie} and \textbf{production\_countries}. In BCNF, see
		\textbf{movie\_genre} for explanation.
	\item{\textbf{production\_companies :}}
		Contains all production companies and their respective id's that
		where encountered in the \textbf{production\_companies} row of
		csv file. Similar to \textbf{genre} also in BCNF.
	\item{\textbf{movie\_production\_companies :}}
		Relationship table for the many-to-many relationship between
		\textbf{movie} and \textbf{production\_companies}. Like all
		other relationship tables it is also in BCNF.
\end{enumerate}

\pagebreak
\section{Queries}
\begin{enumerate}
	\item
		\begin{lstlisting}
		SELECT avg(budget) FROM movie;
		\end{lstlisting}
  		\begin{tabular}{| c | }
    			\hline
    				29045039.8753 \\
    			\hline
  		\end{tabular}

	\item
		\begin{lstlisting}
		SELECT title, company
		FROM
	  	  (
	    	    SELECT title, company, iso_3166_1
	    	    FROM movie
	    	    INNER JOIN movie_production_countries
	    	    ON movie.id=movie_production_countries.movie_id
	    	    INNER JOIN movie_production_companies
	    	    ON movie_production_companies.movie_id=movie_production_countries.movie_id
	    	    INNER JOIN production_companies company
	    	    ON movie_production_companies.company_id = company.id
	  	  ) AS full
		WHERE full.iso_3166_1='US';
		\end{lstlisting}
		\begin{tabular}[t]{| l | r | }
			\hline
			Four Rooms & Miramax Films \\ \hline
			Four Rooms & A Band Apart \\ \hline
			Star Wars & Lucasfilm \\ \hline
			Star Wars & Twentieth Century Fox Film Corporation \\ \hline
			Finding Nemo & Pixar Animation Studios \\
			\hline
		\end{tabular}
	\item
		\begin{lstlisting}
		SELECT title, revenue
		FROM movie
		ORDER BY revenue DESC LIMIT 5;
		\end{lstlisting}
		\begin{tabular}[t]{| l | r | }
			\hline
			Avatar & 2787965087 \\ \hline
			Titanic & 1845034188 \\ \hline
			The Avengers & 1519557910 \\ \hline
			Jurassic World & 1513528810 \\ \hline
			Furious 7 & 1506249360 \\
			\hline
		\end{tabular}
	\pagebreak
	\item
		\begin{lstlisting}
                SELECT mys.title, genre.name
                FROM
                  (
                      SELECT title, g2.name, g.movie_id
                      FROM movie
                        INNER JOIN movie_genre g ON movie.id = g.movie_id
                        INNER JOIN genre g2 ON g.genre_id = g2.id
                      WHERE g2.name="Science Fiction"
                  ) AS sci
                  INNER JOIN
                    (
                      SELECT title, g2.name, g.movie_id
                      FROM movie
                        INNER JOIN movie_genre g ON movie.id = g.movie_id
                        INNER JOIN genre g2 ON g.genre_id = g2.id
                      WHERE g2.name = "Mystery"
                    ) AS mys
                  ON sci.movie_id=mys.movie_id
                  INNER JOIN movie_genre ON movie_genre.movie_id=mys.movie_id
                  INNER JOIN genre ON genre.id=movie_genre.genre_id;
		\end{lstlisting}
		\begin{tabular}[t]{| l | r | }
			\hline
			Tomorrowland & Adventure \\ \hline
			Tomorrowland & Family \\ \hline
			Tomorrowland & Mystery \\ \hline
			Tomorrowland & Science Fiction \\ \hline
			Inception & Action \\
			\hline
		\end{tabular}
	\item
		\begin{lstlisting}
		SELECT title, popularity
		FROM movie
		WHERE popularity > (SELECT avg(popularity) FROM movie);
		\end{lstlisting}
		\begin{tabular}[t]{| l | r | }
			\hline
			Four Rooms & 22.87623 \\ \hline
			Star Wars & 126.393695 \\ \hline
			Finding Nemo & 85.688789 \\ \hline
			Forrest Gump & 138.133331 \\ \hline
			American Beauty & 80.878605 \\
			\hline
		\end{tabular}



\end{enumerate}



\end{document}
